\documentclass{article}

\usepackage{fancyhdr}
\usepackage{extramarks}
\usepackage{amsmath}
\usepackage{amsthm}
\usepackage{amsfonts}
\usepackage{tikz}
\usepackage{forest}
\usepackage{mathtools}
\usepackage[plain]{algorithm}
\usepackage{algpseudocode}

\usetikzlibrary{automata,positioning}
\forestset{
  circles/.style={
    for tree={
      math content,
      circle,
      draw,
      text width=1em,
      text centered,
      edge=->,
      s sep'+=5pt, % increase distance between siblings by 5pt
      l sep'+=5pt, % increase distance between levels by 5pt
    },
    before typesetting nodes={
      for tree={% make circles uniform in size
        content/.wrap value=\strut ##1,
        % split content of nodes into content and value for my label
        split option={content}{:}{content,my label},
      },
    },
  },
  my label/.style={% put the label left or right, depending on which child and level we have
    label/.process={On=On=|? {n'}{1} {level}{0} {45:$#1$}{135:$#1$}},
  },
}

%
% Basic Document Settings
%

\topmargin=-0.45in
\evensidemargin=0in
\oddsidemargin=0in
\textwidth=6.5in
\textheight=9.0in
\headsep=0.25in

\linespread{1.1}

\pagestyle{fancy}
\lhead{\hmwkAuthorName}
\chead{\hmwkClass\ (\hmwkClassInstructor\ \hmwkClassTime): \hmwkTitle}
\rhead{\firstxmark}
\lfoot{\lastxmark}
\cfoot{\thepage}

\renewcommand\headrulewidth{0.4pt}
\renewcommand\footrulewidth{0.4pt}

\setlength\parindent{0pt}

%
% Create Problem Sections
%

\newcommand{\enterProblemHeader}[1]{
    \nobreak\extramarks{}{Problem \arabic{#1} continued on next page\ldots}\nobreak{}
    \nobreak\extramarks{Problem \arabic{#1} (continued)}{Problem \arabic{#1} continued on next page\ldots}\nobreak{}
}

\newcommand{\exitProblemHeader}[1]{
    \nobreak\extramarks{Problem \arabic{#1} (continued)}{Problem \arabic{#1} continued on next page\ldots}\nobreak{}
    \stepcounter{#1}
    \nobreak\extramarks{Problem \arabic{#1}}{}\nobreak{}
}

\setcounter{secnumdepth}{0}
\newcounter{partCounter}
\newcounter{homeworkProblemCounter}
\setcounter{homeworkProblemCounter}{1}
\nobreak\extramarks{Problem \arabic{homeworkProblemCounter}}{}\nobreak{}

%
% Homework Problem Environment
%
% This environment takes an optional argument. When given, it will adjust the
% problem counter. This is useful for when the problems given for your
% assignment aren't sequential. See the last 3 problems of this template for an
% example.
%
\newenvironment{homeworkProblem}[1][-1]{
    \ifnum#1>0
        \setcounter{homeworkProblemCounter}{#1}
    \fi
    \section{Problem \arabic{homeworkProblemCounter}}
    \setcounter{partCounter}{1}
    \enterProblemHeader{homeworkProblemCounter}
}{
    \exitProblemHeader{homeworkProblemCounter}
}

%
% Homework Details
%   - Title
%   - Due date
%   - Class
%   - Section/Time
%   - Instructor
%   - Author
%

\newcommand{\hmwkTitle}{Assignment\ \#3}
\newcommand{\hmwkDueDate}{February 12, 2014}
\newcommand{\hmwkClass}{Information Theory}
\newcommand{\hmwkClassTime}{}
\newcommand{\hmwkClassInstructor}{Marek Smieja}
\newcommand{\hmwkAuthorName}{\textbf{Szymon Maszke}}

%
% Title Page
%

\title{
    \vspace{2in}
    \textmd{\textbf{\hmwkClass:\ \hmwkTitle}}\\
    \normalsize\vspace{0.1in}\small{Due\ on\ \hmwkDueDate\ at 3:10pm}\\
    \vspace{0.1in}\large{\textit{\hmwkClassInstructor\ \hmwkClassTime}}
    \vspace{3in}
}

\author{\hmwkAuthorName}
\date{}

\renewcommand{\part}[1]{\textbf{\large Part \Alph{partCounter}}\stepcounter{partCounter}\\}

%
% Various Helper Commands
%

% Useful for algorithms
\newcommand{\alg}[1]{\textsc{\bfseries \footnotesize #1}}

% For derivatives
\newcommand{\deriv}[1]{\frac{\mathrm{d}}{\mathrm{d}x} (#1)}

% For partial derivatives
\newcommand{\pderiv}[2]{\frac{\partial}{\partial #1} (#2)}

% Integral dx
\newcommand{\dx}{\mathrm{d}x}

% Alias for the Solution section header
\newcommand{\solution}{\textbf{\large Solution}}

% Probability commands: Expectation, Variance, Covariance, Bias
\newcommand{\E}{\mathrm{E}}
\newcommand{\Var}{\mathrm{Var}}
\newcommand{\Cov}{\mathrm{Cov}}
\newcommand{\Bias}{\mathrm{Bias}}

\begin{document}

  \maketitle

  \pagebreak

  \begin{homeworkProblem}
    Prove that Shannon coding is prefix.\\\\
    \solution\\

    \proof A b c
  \end{homeworkProblem}

\pagebreak

\begin{homeworkProblem}
  Let $A= \{0,1\}$ be the source alphabet and $P=\{0.1, 0.9\}$ it's probability
  distribution.
  \begin{itemize}
    \item Find Huffman and Shannon codes for all pairs, triplets and
      quadruplets of source alphabet.
    \item Compare their estimated code lengths attributable on symbol with entropy
      source
    \item Consider distribution $Q = {0.4, 0.6}$ and repeat the task
  \end{itemize}
  \solution\\\\
  \part\\

  Let's calculate probabilties for each pair and triplet
  \begin{itemize}
    \item pairs: $[00,01,10,11]$ described by probability vector: $[0.01, 0.09,
      0.09, 0.81]$.
    \item triplets: $[000, 001, 010, 011, 100, 101, 110,111]$ described by
      probability vector:\\ $[0.001, 0.009, 0.009, 0.081, 0.009, 0.081, 0.081,
      0.72]$
  \end{itemize}
  \part\\

  Sort symbols in decreasing order based on their probability, calculate
  length of each code using the formula:
  \begin{equation}
  l_i = \left \lceil -\log_2(p_i) \rceil \right
  \end{equation}
  and calculate their cumulative probabilities using formula:
  \begin{equation}
    P_c = \sum_{i<k} p(i)
  \end{equation}
  \begin{itemize}
    \item $[11,10,01,00]$ with $p_p = [0.81,0.09, 0.09, 0.01]$, $l_p =
      [1,4,4,7]$ and $P_c_p = [0, 0.81, 0.90, 0.99]$
    \item Analogous for triplets...
  \end{itemize}
  To receive the code we have to take first $l_k$ digits from cumulative
  probability transformed to binary digits, so we receive:
  \begin{itemize}
    \item Binary cumulative probability: $P_c_p_b = [0.0, 0.1100, 0.1110,
      0.1111111]$, which gives us the code: $[11:0, 10:1100, 01:1110,
      00:1111111]$
    \item Analogous for triplets...
  \end{itemize}
  Source entropy of pairs is given by:
  \begin{equation}
    \sum_{i}p_i \log_2 p_i \leq h(X) + 1
  \end{equation}
  Average code length for Shannon coding is given by:
  \begin{equation}
    \sum_{i}p_i \left \lceil \log_2 p_i \rceil \right \leq h(X) + 1
  \end{equation}
  , with the sum being equal to element-wise vector multiplication of $l_p$ and
  $p_p$.\\
  Applying above formulas for pairs gives us:
  \begin{align}
    0.81+2*0.36+0.07 < 0.81*0.246 + 2*0.312 + 0.066 + 1\\
    1.6 < 0.88926 + 1
  \end{align}
  which is consistent with the approximation for average length of Shannon
  coding.
  \pagebreak

  \part\\

  \textbf{Introduction:}\\
  Huffman coding is optimal based on average code length (in contrast to Shannon
  coding). Based on binary trees with leaves representing symbols and and path
  from the root to leaves their codes (respectively).\\

  \solution\\

  To create Huffman code we have to use recursive approach for binary tree
  creation. Starting with two smallest probabilities and differentiating between
  them only with one value, we climb up the probability 'ladder'. For each pair,
  we sum their probability and apply it back to the list.\\

  Our sorted probability list for pairs is: $p_p = [0.81,0.09, 0.09, 0.01]$ and
  appropriate symbols are: $SYMBOLS = [11, 01, 10, 00]$.\\
  We take two smallest values to create first binary tree.
  It will be parametrized by probability and appropriate code (root will be
  given code placeholder: *)

  \begin{figure}[!h]
  \centering
  \begin{forest}
    circles
    [0.1:*
    [.09:0]
    [.01:1]
    ]
  \end{forest}
  \end{figure}
  Our current list consists of following probabilities: $p_p = [0.81,0.10,0.9]$ and
  unified symbols:\\ $SYMBOLS = [11, *, 01]$, where $*$ is unified $00$ and $10$
  \begin{figure}[!h]
  \centering
  \begin{forest}
    circles
    [.19:*
      [.09:0]
      [0.1:*
      [.09:0]
      [.01:1]
      ]
    ]
  \end{forest}
  \end{figure}

  \pagebreak
  Our current list consists of following probabilities: $p_p = [0.81,0.19]$ and
  unified symbols:\\ $SYMBOLS = [11, *]$
  \begin{figure}[!h]
  \centering
  \begin{forest}
    circles
    [1:*
      [.19:*
        [.09:0]
        [0.1:*
        [.09:0]
        [.01:1]
        ]
      ]
      [.81:1]
    ]
  \end{forest}
  \end{figure}

  Probability equal to one finishes the algorithm and we can code each symbol
  (moving to the left applies zero, while moving to the right applies one):
  \begin{equation}
    [11,01,10,00] = [1,00,010,011]
  \end{equation}
  Based on the coding above $l_p = [1,2,3,3]$ and probabilities for each code is
  respectively: $p_p = [0.81, 0.09,0.09,0.01]$, so average code length is:
  \begin{align}
    0.81+5*0.09++0.01*3 \leq h(X) + 1\\
    1.29 \leq 0.88926 + 1
  \end{align}
  It should be noted, that average code length for Huffman coding is smaller
  than the one found using Shannon coding with $1.29$ and $1.6$ respectively,
  hence can be considered more optimal with respect to this attribute.

  \LARGE FINISH OTHER EXAMPLES SOMEDAY...
\end{homeworkProblem}
\pagebreak

\begin{homeworkProblem}
  Code text `alabla' using arithmetic coding. Estimate probability distribution
  from data.\\

  \solution\\

  \part

  First we have to estimate probability distribution of a given string, as the
  task is trivial I will only provide a dictionary containing it:
  \begin{equation}
    [a:0.5, l:0.(3), b:0.1(6)]
  \end{equation}

  \part\\

  To code the string we have to use cumulative probability, e.g.:
  \begin{equation}
    \sum_{i<k} p_i = [0, 0.5, 0.83, 1]
  \end{equation}
  \begin{itemize}
    \item We start by coding first letter `a' giving it $I_0 = [0,0.5]$
    \item For letter l: $I_1 = 0 + 0.5 * [0.5, 0.83] = [0.25, 0.415]$
    \item For letter a: $I_2 = 0.25 + 0.165 * [0, 0.5] = [0.25, 0.3325]$
    \item For letter b: $I_3 = 0.25 + 0.08325 * [0.83, 1] = [0.3190975, 0.33325]$
    \item For letter l: $I_4 = 0.3190975 + 0.0141525 * [0.5, 0.83] = [0.32617375, 0.330844075]$
    \item For letter a: $I_5 = 0.3261737 + 0.004670325 * [0, 0.5] = [0.32617375, 0.3285088625]$
  \end{itemize}
  Result of this coding is the interval $[0.32617375, 0.3285088625]$, the marker
  can be given by it's mean: $\textbf{z = 0.32734130625}$ and length of the word
  $\textbf{n=6}$\\

  \part

  \LARGE Add decoding if you have time

\end{homeworkProblem}
\pagebreak

% Non sequential homework problems
%
% Jump to problem 18
\begin{homeworkProblem}[18]
\end{homeworkProblem}

% Continue counting to 19

\end{document}
