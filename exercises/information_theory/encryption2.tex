\documentclass{article}

\usepackage{fancyhdr}
\usepackage{extramarks}
\usepackage{amsmath}
\usepackage{amsthm}
\usepackage{amsfonts}
\usepackage{tikz}
\usepackage[plain]{algorithm}
\usepackage{algpseudocode}
\usepackage{forest}
\usepackage{mathtools}
\usepackage{enumerate}
\usepackage{physics}
\usepackage{amssymb}
% \usetikzlibrary{graphs,graphdrawing,arrows.meta}
% \usegdlibrary{trees}

\usetikzlibrary{automata,positioning}
\forestset{
  circles/.style={
    for tree={
      math content,
      circle,
      draw,
      text width=1em,
      text centered,
      edge=->,
      s sep'+=5pt, % increase distance between siblings by 5pt
      l sep'+=5pt, % increase distance between levels by 5pt
    },
    before typesetting nodes={
      for tree={% make circles uniform in size
        content/.wrap value=\strut ##1,
        % split content of nodes into content and value for my label
        split option={content}{:}{content,my label},
      },
    },
  },
  my label/.style={% put the label left or right, depending on which child and level we have
    label/.process={On=On=|? {n'}{1} {level}{0} {45:$#1$}{135:$#1$}},
  },
}

%
% Basic Document Settings
%

\topmargin=-0.45in
\evensidemargin=0in
\oddsidemargin=0in
\textwidth=6.5in
\textheight=9.0in
\headsep=0.25in

\linespread{1.1}

\pagestyle{fancy}
\lhead{\hmwkAuthorName}
\chead{\hmwkClass\ (\hmwkClassInstructor\ \hmwkClassTime): \hmwkTitle}
\rhead{\firstxmark}
\lfoot{\lastxmark}
\cfoot{\thepage}

\renewcommand\headrulewidth{0.4pt}
\renewcommand\footrulewidth{0.4pt}

\setlength\parindent{0pt}

%
% Create Problem Sections
%

\newcommand{\enterProblemHeader}[1]{
    \nobreak\extramarks{}{Problem \arabic{#1} continued on next page\ldots}\nobreak{}
    \nobreak\extramarks{Problem \arabic{#1} (continued)}{Problem \arabic{#1} continued on next page\ldots}\nobreak{}
}

\newcommand{\exitProblemHeader}[1]{
    \nobreak\extramarks{Problem \arabic{#1} (continued)}{Problem \arabic{#1} continued on next page\ldots}\nobreak{}
    \stepcounter{#1}
    \nobreak\extramarks{Problem \arabic{#1}}{}\nobreak{}
}
\setcounter{secnumdepth}{0}
\newcounter{partCounter}
\newcounter{homeworkProblemCounter}
\setcounter{homeworkProblemCounter}{1}
\nobreak\extramarks{Problem \arabic{homeworkProblemCounter}}{}\nobreak{}

%
% Homework Problem Environment
%
% This environment takes an optional argument. When given, it will adjust the
% problem counter. This is useful for when the problems given for your
% assignment aren't sequential. See the last 3 problems of this template for an
% example.
%
\newenvironment{homeworkProblem}[1][-1]{
    \ifnum#1>0
        \setcounter{homeworkProblemCounter}{#1}
    \fi
    \section{Problem \arabic{homeworkProblemCounter}}
    \setcounter{partCounter}{1}
    \enterProblemHeader{homeworkProblemCounter}
}{
    \exitProblemHeader{homeworkProblemCounter}
}

%
% Homework Details
%   - Title
%   - Due date
%   - Class
%   - Section/Time
%   - Instructor
%   - Author
%

\newcommand{\hmwkTitle}{Assignment\ \#2}
\newcommand{\hmwkDueDate}{February 12, 2014}
\newcommand{\hmwkClass}{Information Theory}
\newcommand{\hmwkClassTime}{}
\newcommand{\hmwkClassInstructor}{Marek Smieja}
\newcommand{\hmwkAuthorName}{\textbf{Szymon Maszke}}

%
% Title Page
%

\title{
    \vspace{2in}
    \textmd{\textbf{\hmwkClass:\ \hmwkTitle}}\\
    \normalsize\vspace{0.1in}\small{Due\ on\ \hmwkDueDate\ at 3:10pm}\\
    \vspace{0.1in}\large{\textit{\hmwkClassInstructor\ \hmwkClassTime}}
    \vspace{3in}
}

\author{\hmwkAuthorName}
\date{}

\renewcommand{\part}[1]{\textbf{\large Part \Alph{partCounter}}\stepcounter{partCounter}\\}

%
% Various Helper Commands
%

% Useful for algorithms
\newcommand{\alg}[1]{\textsc{\bfseries \footnotesize #1}}

% For derivatives
\newcommand{\deriv}[1]{\frac{\mathrm{d}}{\mathrm{d}x} (#1)}

% For partial derivatives
\newcommand{\pderiv}[2]{\frac{\partial}{\partial #1} (#2)}

% Integral dx
\newcommand{\dx}{\mathrm{d}x}

% Alias for the Solution section header
\newcommand{\solution}{\textbf{\large Solution}}

% Probability commands: Expectation, Variance, Covariance, Bias
\newcommand{\E}{\mathrm{E}}
\newcommand{\Var}{\mathrm{Var}}
\newcommand{\Cov}{\mathrm{Cov}}
\newcommand{\Bias}{\mathrm{Bias}}

\begin{document}

  \maketitle

  \pagebreak

\begin{homeworkProblem}
  Give one example of encoding from each group mentioned in lectures:\\\\
  \solution\\
  \begin{itemize}
    \item \textbf{Non non-singular encoding} (coding function, each symbol can be
      encoded in any way [resulting codes can be equal]\\
      \textbf{Example}:
      \begin{align}
        S = \{a,b,c,d\}, A=\{0\}, \phi: S \rightarrow A* \\
        \phi(a) = 0\\
        \phi(b) = 0\\
        \phi(c) = 0\\
        \phi(d) = 0
      \end{align}
    \item \textbf{Non-singular encoding} (injective coding function [resulting codes
      for each symbol cannot be equal]), fulfilling the property:
      \begin{align}
        s_1 \neq s_2 \implies \phi(s_1) \neq \phi(s_2)
      \end{align}
      \textbf{Example}:
      \begin{align}
        S = \{a,b,c,d\}, A=\{0,1,2\}, \phi: S \rightarrow A* \\
        \phi(a) = 0\\
        \phi(b) = 1\\
        \phi(c) = 10\\
        \phi(d) = 2
      \end{align}
      \textbf{Codes do not have to be uniquely decodable!}
    \item \textbf{Fixed length codes} (same number of bits used for each
      encoding)\\
      \textbf{Example} (no compression, uniquely decodable):
      \begin{align}
        S = \{a,b\}, A=\{0,1\}, \phi: S \rightarrow A* \\
        \phi(a) = 1\\
        \phi(b) = 0
      \end{align}
    \item \textbf{Uniquely decodable} (Code is uniquely decodable if there is
      only one series able to produce it)\\
      \textbf{Example}: above\\
      \textbf{Definitions}: Code is uniquely decodable if it's extensions is
      non-singular, namely:
      \begin{align}
        \phi(s_1,s_2,\ldots,s_k) := \phi(s_1)\phi(s_2)\ldots\phi(s_k)
      \end{align}
      \pagebreak
    \item \textbf{Prefix codes} (None of the codes are the prefix of other
      codes)\\
      \begin{align}
        S = \{a,b,c,d\}, A=\{0,1\}, \phi: S \rightarrow A* \\
        \phi(a) = 00\\
        \phi(b) = 10\\
        \phi(a) = 11\\
        \phi(b) = 01
      \end{align}

  \end{itemize}
\end{homeworkProblem}

\begin{homeworkProblem}
  Assume $m$-element coding alphabet. We want uniquely decodable code with
  lengths $l = (1,1,2,3,2,3)$. What is the minimum value of $m$ we can choose?
  Find encoding for this length.\\\\
  \solution\\\\
  \part\\\\
  To resolve this problem we have to use \textbf{Kraft's inequality} given by
  following equation:
  \begin{align}
    \sum_{i=1}^{r}m^{-li} \leq 1
  \end{align}
  where:
  \begin{itemize}
    \item $r$ count of elements in coding alphabet A
    \item $l_i$ length of each element in coding alphabet A
    \item $m$ size of coding alphabet (and sought variable)
  \end{itemize}

  Substituting our example to above equation gives us:
  \begin{align}
    \frac{2}{m^2} + \frac{2}{m^3} + \frac{2}{m^4} \leq 1\\
  \end{align}
  Solving above unequality for m gives us $\textbf{m = 2.919}$. Size of the
  coding alphabet can only take integer numbers, hence:
  \begin{align}
  m = \left \lceil{2.919}\right \rceil = 3
  \end{align}
  \pagebreak

  \part\\\\
To create prefix code with minimal size of coding alphabet, we have to draw tree
and apply codes from the left side (starting with the shortest ones)

\begin{figure}[!h]
\centering
\begin{forest}
  circles
  [:root
  [x_1:0]
  [x_2:1]
  [:2
    [x_3:0]
    [x_4:1]
    [:2
      [x_5:0]
      [x_6:1]
    ]
  ]
  ]
\end{forest}
\end{figure}

Obtained codes from the tree are given by the following vector:
\begin{align}
  \textbf{[0, 1, 20, 21, 220, 221]}
\end{align}
\end{homeworkProblem}

\pagebreak

% Non sequential homework problems
%
% Jump to problem 18
\begin{homeworkProblem}[4]
  Check continuity and differentiability of the function:
  \begin{equation}
    sh(x) =
    \begin{cases*}
      0 & if $x=0$ \\
      -x log x & otherwise
    \end{cases*}
  \end{equation}
  \solution\\\\
  With the following function we have one point suspect od discontinuity, namely
  $\textbf{0}$, so we have to check limes from left and right side. \\\\
  \part

  \begin{enumerate}
    \item Transform $-x log x$ to $-\frac{log x}{x^{-1}}$ so we can use l'Hospital
      rule
    \item
      \begin{equation}
        \lim_{x \to 0} -\frac{log x}{x^{-1}}
        \overset{H}{\underset{\left[\frac{0}{0}\right]}{=}} \lim_{x \to
        0^+}\frac{x^{-1}}{x^{-2}} = \lim_{x \to
      0^+}x^{-1}*x^{2} = \lim_{x \to 0^+} x = 0
      \end{equation}
    \item
      \begin{equation}
        \lim_{x \to 0} -\frac{log x}{x^{-1}}
        \overset{H}{\underset{\left[\frac{0}{0}\right]}{=}} \lim_{x \to
        0^-}\frac{x^{-1}}{x^{-2}} = \lim_{x \to
      0^-}x^{-1}*x^{2} = \lim_{x \to 0^-} x = 0
      \end{equation}
    \item Left and right side limes is equal to zero and $sh(x)$ in zero is
      equal to zero, therefore the function is continuous and differentiable
  \end{enumerate}

\end{homeworkProblem}
\begin{homeworkProblem}
  Find maximum of the function $sh(x) + sh(1-x)$ with $x \in [0,1]$ and draw
  it.\\\\
  \solution\\\\
  First of all we need the derivative of $sh(x) + sh(1-x)$, which is:
  \begin{equation}
    \deriv{sh(x)} + \deriv{sh(1-x)} =
    \begin{cases*}
      0 & if $x=0$ \\
      -x logx f''(x)-(log(x)+1)f'(x)+log(1-x)+1& otherwise
    \end{cases*}
  \end{equation}
  Comparing derivative to zero gives us:
  \begin{align*}
    -x logx f''(x)-(log(x)+1)f'(x)+log(1-x)+1 = 0\\
    x=\frac{1}{2} - \frac{\sqrt{e^2 - 4}}{2e}
  \end{align*}
  And nobody in their right mind will calculate all of it by hand.
\end{homeworkProblem}
\pagebreak

\begin{homeworkProblem}
  There is an event $X=x$ occuring with probability $p_x = P(X=x)$.
  Let $I(p(x)) = -\log_2p(x)$ be the amount of information in event. \\
  We play a game, where we choose one integer $(x)$ number in the range $[0,63]$.
  Opponent has to guess foremenetioned integer asking questions with only two
  possible answers (Yes or No).
  \begin{itemize}
    \item How many questions have to be asked to know the value of X?
    What's the information contained in the answer for each of those questions?
    What's the amount of information we get after asking all of the questions?
    \item Assume opponent is asking questions "Whether this number is y?", where
      y ia an integer we didn't ask for yet. We guessed the value of x after
      asking n questions. What is the information contained in each of those answers?
      Consider both positive and negative answers.
    \item Calculate the information received after n questions
  \end{itemize}
  \solution\\\\
  \part

  We have to make following assumption to give meaningful answers:\\\\
  \textbf{Assumption:} Choice of each integer in the range (0,63) is equally likely and
  follows uniform distribution (maximum entropy)\\

  For easier comprehension I will introduce two easy notations.
  \begin{itemize}
    \item Maximum amount of information will be given for the smallest
      probability for random variable, e.g.
      \begin{equation}
        \lim_{p(i) \to 0^+} -\log_2p(i) = \infty
      \end{equation}
    \item Smallest amount of information is given by:
      \begin{equation}
        \lim_{p(i) \to \infty} -\log_2p(i) = 0
      \end{equation}
    \item Information about X is equal to it's entropy, e.g.:
      \begin{equation}
        I(X;X) = H(X)
      \end{equation}
  \end{itemize}

  \part

  Solutions to the problems:
  \begin{itemize}
    \item Possible answers (`Yes' or `No') follow Bernoulli distribution, hence
      maximum entropy is given by $p(y) = \frac{1}{2}$, where y is our answer.
      The only possibility for $p(y) = \frac{1}{2}$ is when both answers are
      Which gives us $x=0$ and $y=0$
      equally likely. With set consisting of 64 numbers we can divide it into
      two parts with 32 elements each by asking question: `Is the number bigger
      than $a$`, where $a$ is the center value of the set (in this case 31).
      To certainly receive the answer for set with $n$ sorted values and
      asking binary questions (which allow to divide the set) we have to ask:
      \begin{equation}
        \log_2(n)
      \end{equation}
      questions (\textbf{6 in this case}).\\
      For those six questions we get a total information amount received equal
      to six as well:
      \begin{equation}
        6* \log_2(\frac{1}{2}) = 6
      \end{equation}
      Which sums up to total count of questions we have to ask.
      \pagebreak

    \item If opponent guesses the answer right away, the amount of information
      he receives is equal to:
      \begin{equation}
        - \log_2(\frac{1}{64}) = 6
      \end{equation}
      With each answer `No' amount of information received drops by
      $\frac{\log(\frac{n-1}{n})}{\log(2)}$, where $n$ is the size of set for
      current guess, and $n-1$ the size of set for the guess before.\\\\
      Same situation applies to all negative guesses with sudden `Yes' followed
      by it.

    \item For information contained in guess after $n$ tries we can provide the generic formula:
      \begin{equation}
        - \log_2(\frac{1}{m-n})
      \end{equation}
      where $m$ is the count of set in the beginning,
      given the answers are binary and integer's probability ($p(x)$) follows
      the uniform distribution.
  \end{itemize}

\end{homeworkProblem}
\begin{homeworkProblem}
  Calculate extrema of $z = x+y$ on $x^2+y^2 = 1$\\\\
  \solution
  \begin{itemize}
    \item Let's calculate partial derivatives of the function:
      \begin{align}
        \pdv{z}{x}(x+y) = 1\\
        \pdv{z}{y}(x+y) = 1
      \end{align}
      which gives us statnioray point $P = (1,1)$ where the extremum might be
      located.
    \item Second order derivatives would return 0 and so would $det$ for the
      matrix $W$ so this test is inconclusive
    \item In this scenario we can pose it as optimization problem with Lagrangian:
      \begin{align}
        \mathcal{L}({x,y,\lambda}) = f(x,y) - \lambda  g(x,y)\\
        \mathcal{L}(x,y,\lambda) = x+y - \lambda(x^2+y^2 - 1) = -\lambda x^2 -
        \lambda y^2 + x + y -\lambda
      \end{align}
    \item Now we have to calculate partial derivatives with respect to each of
      the parameters:
      \begin{align}
        \pdv{\mathcal{L}}{x} = -2 x \lambda +1\\
        \pdv{\mathcal{L}}{y} = -2 y \lambda +1\\
        \pdv{\mathcal{L}}{\lambda} = -x^2 -y^2 - 1
      \end{align}
    \item Creating system of equations with each equal to zero gives us those
      values:
      \begin{align}
        x = - \frac{i}{\sqrt{2}} \land y = -\frac{i}{\sqrt{2}}\\
        \lor \\
        x = \frac{i}{\sqrt{2}} \land y = \frac{i}{\sqrt{2}}
      \end{align}
  \end{itemize}
\end{homeworkProblem}

% Continue counting to 19

\end{document}
